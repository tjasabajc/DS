\documentclass{beamer}

\usepackage[slovene]{babel}
\usepackage{amsfonts,amssymb}
\usepackage[utf8]{inputenc}
\usepackage{lmodern}
\usepackage[T1]{fontenc}

\usetheme{Warsaw}

\def\N{\mathbb{N}} % mnozica naravnih stevil
\def\Z{\mathbb{Z}} % mnozica celih stevil
\def\Q{\mathbb{Q}} % mnozica racionalnih stevil
\def\R{\mathbb{R}} % mnozica realnih stevil
\def\C{\mathbb{C}} % mnozica kompleksnih stevil


\def\qed{$\hfill\Box$}   % konec dokaza
\newtheorem{izrek}{Izrek}
\newtheorem{trditev}{Trditev}
\newtheorem{posledica}{Posledica}
\newtheorem{lema}{Lema}
\newtheorem{definicija}{Definicija}
\newtheorem{pripomba}{Pripomba}
\newtheorem{primer}{Primer}
\newtheorem{zgled}{Zgled}
\newtheorem{zgledi}{Zgledi uporabe}
\newtheorem{zglediaf}{Zgledi aritmetičnih funkcij}
\newtheorem{oznaka}{Oznaka}

\title{Geometrijska interpolacija štirih točk s parabolično krivuljo}
\author{Tjaša Bajc}
\institute{mentorica \\ izr.~prof.~dr.~Marjetka Knez}

\date{3.\ april 2017}

\begin{document}


%%%%%%%%%%%%%%%%%%%%%%%%%%%%%%%%%%%%%%%%%%%%%%%%%%%%%%%%%%%%%%%%%%%%%

\begin{frame}
\titlepage
\end{frame}

%%%%%%%%%%%%%%%%%%%%%%%%%%%%%%%%%%%%%%%%%%%%%%%%%%%%%%%%%%%%%%%%%%%%%

%\begin{frame}
%\frametitle{Opis problema}
%
%Vprašanje je, ali obstaja parabolična krivulja, ki poteka skozi dane štiri točke v ravnini. Če obstaja, nas zanima, koliko je takih krivulj.
%
%Problema se bomo lotili na dva načina.
%
%\begin{itemize}
%
%\item \textbf{Geometrijski pristop} \\
%Obravnavamo lik, katerega oglišča so dane točke.
%
%\item \textbf{Interpolacija} \\
% Pomagali si bomo z razvojem po Lagrangeevih baznih polinomih.
%
%\end{itemize}
%
%\end{frame}


%%%%%%%%%%%%%%%%%%%%%%%%%%%%%%%%%%%%%%%%%%%%%%%%%%%%%%%%%%%%%%%%%%%%%

\begin{frame}
\frametitle{Geometrijska interpolacija}

Opazujemo štirikotnik, ki ga tvorijo dane štiri točke. Od lastnosti tega štirikotnika je odvisno, ali točke lahko interpoliramo s parabolično krivuljo.

\begin{izrek}
Naj bo $T = \{ T_0, T_1, T_2, T_3 \}$ nabor štirih točk, od katerih nobene tri niso kolinearne.

\begin{enumerate}[i)]
\pause
\item Če so točke iz $T$ oglišča konkavnega štirikotnika, danih točk ne moremo interpolirati s parabolično krivuljo.
\pause
\item Če so točke iz $T$ oglišča paralelograma, danih točk ne moremo interpolirati s parabolično krivuljo.
\pause
\item Če so točke iz $T$ oglišča trapeza, ki ni paralelogram, lahko dane točke interpoliramo z natanko eno parabolično krivuljo.
\pause
\item Če so točke iz $T$ oglišča konveksnega štirikotnika, ki ni trapez, lahko dane točke interpoliramo z natanko dvema paraboličnima krivuljama.
\end{enumerate}

\end{izrek}

\end{frame}


%%%%%%%%%%%%%%%%%%%%%%%%%%%%%%%%%%%%%%%%%%%%%%%%%%%%%%%%%%%%%%%%%%%%%

%\begin{frame}
%\frametitle{Lagrangeevi bazni polinomi}
%
%Lagrangeevi bazni polinomi tvorijo bazo za prostor polinomov določene stopnje.
%
%\pause
%Oglejmo si polinome $ \ell_{0,2}(t), \ell_{1,2}(t) $ in $ \ell_{2,2}(t) $, ki so baza za prostor polinomov druge stopnje.
%
%\begin{definicija}
%
%	$$ \ell_{0,2}(t) =  \frac{(t - t_1)(t - t_2)}{(t_0 - t_1)(t_0 - t_2)} $$
%	$$ \ell_{1,2}(t) =  \frac{(t - t_0)(t - t_2)}{(t_1 - t_0)(t_1 - t_2)} $$
%	$$ \ell_{2,2}(t) =  \frac{(t - t_0)(t - t_1)}{(t_2 - t_0)(t_2 - t_1)} $$
%
%\end{definicija}
%
%\end{frame}

%%%%%%%%%%%%%%%%%%%%%%%%%%%%%%%%%%%%%%%%%%%%%%%%%%%%%%%%%%%%%%%%%%%%%

\begin{frame}

\frametitle{Paralelogram}

%\begin{figure}

%\end{frigure}

\end{frame}

%%%%%%%%%%%%%%%%%%%%%%%%%%%%%%%%%%%%%%%%%%%%%%%%%%%%%%%%%%%%%%%%%%%%%


\end{document}