\documentclass[mat1]{fmfdelo}


% aktivirajte pakete, ki jih potrebujete
\usepackage{enumerate}

% za številske množice uporabite naslednje simbole
\newcommand{\R}{\mathbb R}
\newcommand{\N}{\mathbb N}
\newcommand{\Z}{\mathbb Z}
\newcommand{\C}{\mathbb C}
\newcommand{\Q}{\mathbb Q}
\newcommand{\PP}{\mathbb P}
\newcommand{\B}{\mathbb B}
\newcommand{\al}{\alpha}


% matematične operatorje deklarirajte kot take, da jih bo Latex pravilno stavil
\DeclareMathOperator{\conv}{conv}

% na razpolago so naslednja matematična okolja, ki jih kličemo s parom 
% \begin{imeokolja}[morebitni komentar v oklepaju] ... \end{imeokolja}
%
% definicija, opomba, primer, zgled, lema, trditev, izrek, posledica, dokaz
% 


% vstavite svoje definicije ...
%  \newcommand{}{}


% naslednje ukaze ustrezno napolnite
\avtor{Tjaša Bajc} 

\naslov{Geometrijska interpolacija štirih točk s parabolično krivuljo}
\title{Angleški prevod slovenskega naslova dela}

 \mentorica{izr.~prof.~dr.~Marjetka Knez}

\letnica{2018} % leto diplome

%  V povzetku na kratko opišite vsebinske rezultate dela. Sem ne sodi razlaga organizacije dela --
%  v katerem poglavju/razdelku je kaj, pač pa le opis vsebine.
\povzetek{}

%  Prevod slovenskega povzetka v angleščino. 
\abstract{}

% navedite vsaj eno klasifikacijsko oznako --
% dostopne so na www.ams.org/mathscinet/msc/msc2010.html
\klasifikacija{}
\kljucnebesede{} % navedite nekaj ključnih pojmov, ki nastopajo v delu
\keywords{} % angleški prevod ključnih besed


\begin{document}

%%%%%%%%%%%%%%%%%%%%%%%%%%%%%%%%%%%%%%%%%%%%%%%%%%%%%%%%%%%%%%%%%%%%%%%%%%%%%%%%%
% --------------------------------------------------------------------------------------------------------------------------------------------------------------------------------------------------------------------------
%%%%%%%%%%%%%%%%%%%%%%%%%%%%%%%%%%%%%%%%%%%%%%%%%%%%%%%%%%%%%%%%%%%%%%%%%%%%%%%%%

\section{Uvod}

Imamo štiri točke v ravnini. Zanima nas, kdaj lahko skoznje potegnemo parabolično krivuljo, to je parametrično polinomsko krivuljo stopnje dve, oziroma koliko je takih krivulj. Na primer, če so točke kolinearne, take prave paraboične krivulje očitno ne bomo našli. Kaj pa v ostalih primerih? Izkaže se, da je dovolj opazovati štirikotnik, katerega oglišča so dane točke. Pokazali bomo, da oglišča konveksnega štirikotnika, ki ni trapez, lahko interpoliramo z natanko dvema paraboličnima krivuljama, oglišča trapeza, ki ni paralelogram, pa z natanko eno parabolično krivuljo. V preostalih primerih štirih točk ne moremo interpolirati s parabolično krivuljo.


Poleg geometrijskega pristopa, ki smo ga na kratko povzeli zgoraj, se interpolacije lahko lotimo tudi z Lagrangeevimi baznimi polinomi. Vemo, da lahko štiri točke vedno interpoliramo s kubično krivuljo. Definirali bomo Lagrangeeve polinome stopnje tri in določili pogoje za proste parametre tako, da bo vodilni koeficient interpolacijskega polinoma enak $0$. Tako bomo dobili interpolacijsko polinomsko krivuljo stopnje dve, torej parabolično krivuljo.


V nadaljevanju si bomo ogledali še Hermitov problem. Namesto štirih točk bomo opazovali le dve, v katerih pa bomo poleg vrednosti predpisali tudi smer tangentnega vektorja. Poiskali bomo interpolacijsko krivuljo, ki bo zadoščala danim pogojem. Hermitov problem lahko posplošimo na več točk in opazujemo zlepke, ki jih dobimo z interpolacijo posameznih parov točk. Zlepek, ki ga dobimo na tak način, je geometrijsko zvezna krivulja, ki interpolira dane točke. 

%%%%%%%%%%%%%%%%%%%%%%%%%%%%%%%%%%%%%%%%%%%%%%%%%%%%%%%%%%%%%%%%%%%%%%%%%%%%%%%%%
% --------------------------------------------------------------------------------------------------------------------------------------------------------------------------------------------------------------------------
%%%%%%%%%%%%%%%%%%%%%%%%%%%%%%%%%%%%%%%%%%%%%%%%%%%%%%%%%%%%%%%%%%%%%%%%%%%%%%%%%

\section{Geometrijska interpolacija}

Za začetek definirajmo nekaj pojmov, ki jih bomo potrebovali v nadaljevanju.

\begin{definicija}
Naj bo $A'$ nesingularna $2\times2$ realna matrika, $d, e$ realni števili ter 
$$ A = 
\begin{bmatrix}
A' &
\begin{matrix}
0 \\
0
\end{matrix}
\\
\begin{matrix}
d & e
\end{matrix}
 & 1
\end{bmatrix}
.$$
Matriko $A$ imenujemo \emph{afina matrika}. 
\end{definicija}

% Ali tu rabimo kaj o grupi afinih matrik --- zaprtost za množenje, inverz,...?

Preko afinih matrik vpeljemo ekvivalenčno relacijo na množici realnih simetričnih $3 \times 3$ matrik.

\begin{definicija}
Matriki $B$ in $C$ sta \emph{afino podobni}, če obstaja afina matrika $A$, da velja $B = A C A^{T}$.
Da sta matriki afino podobni, označimo z $B \approx C$.
\end{definicija}

% Ali je to potrebno dokazati? Da je res EKV relacija?

\begin{definicija}
Definirajmo matriko 
$$D = 
\begin{bmatrix}
0 & 0 & 1 \\
0 & -2 & 0 \\
1 & 0 & 0
\end{bmatrix}.
$$
Definirajmo še podmnožico realnih simetričnih $3 \times 3$ matrik $$\PP = \{ B; B \approx d D, d \neq 0 \} .$$
\emph{Parabolična krivulja} je množica točk v ravnini 
\begin{equation}\label{implicitna}
 C_B = \{ (x,y); (x,y,1) B (x, y, 1)^T = 0, B \in \PP \}.
 \end{equation}
\end{definicija}

Če je matrika $B$ enaka $D$, je množica $C_B$ kar $C_B = \{ (x,y); x = y^2 \}$.

\begin{zgled}

(bom pripravila zgled: 

matriko $B$ in sliko parabole, ki jo podaja $C_B$)

\end{zgled}

Parabolično krivuljo lahko podamo v implicitni obliki kot v \eqref{implicitna} ali pa v parametrični obliki. Definirajmo kvadratično parametrizacijo parabolične krivulje, podane s $C_B$.

\begin{definicija}
Naj bodo $p(t)$, $q(t)$ in $r(t) \equiv 1$ linearno neodvisni polinomi stopnje največ dve. Če za nek $B$ iz množice $\PP$ velja 
$$ C_B = \{ (x,y); (x,y,1) B (x, y, 1)^T = 0 \} = \{ (p(t), q(t)); t \in \R \},$$
pravimo, da je $( p(t), q(t), 1)$ \emph{kvadratična parametrizacija} parabolične krivulje $C_B$.
\end{definicija}

Poglejmo, kako bi parametrično krivuljo zapisali v standardni bazi $t^2, t, 1$? Najprej bomo definirali matriko koeficientov, ki povezuje kvadratično parametrizacijo in zapis v standardni bazi, nato pa bomo pokazali, da lahko vsako parabolično krivuljo zapišemo v kvadratični parametrizaciji.

\begin{definicija}
Naj bodo $p(t)$, $q(t)$ in $r(t) \equiv 1$ linearno neodvisni polinomi stopnje največ dve. \emph{Matrika koeficientov $K$} je taka matrika, da velja $ (p(t), q(t), 1) = (t^2,t,1) K$.
\end{definicija} 

\begin{trditev}
Naj bodo $p(t)$, $q(t)$ in $r(t) \equiv 1$ linearno neodvisni polinomi stopnje največ dve, $K$ matrika koeficientov in $B \in \PP$. Tedaj velja
\begin{itemize}
\item $K$ je afina matrika,
\item  $ ( p(t), q(t), 1)$ je kvadratična parametrizacija za $C_B$ natanko tedaj, ko velja $K B K^T = d D$ za neki neničelni d.
\end{itemize}
\end{trditev}

% NOVO %%%%%%%%%%%%%%%%%%%%%%%%%%%%
\begin{dokaz}
Matrika $K$ je nesingularna, ker so $p(t), q(t)$ in 1 linearno neodvisni. Zadnja komponenta vektorja $(p(t), q(t), 1)$ je 1, torej mora biti zadnji stolpec matrike $K$ enak $(0,0,1)^T$. Sledi, da je matrika $K$ afina. Za dokaz druge točke trditve se spomnimo definicije kvadratične parametrizacije.
Velja:
\begin{align*}
0 &= (p(t), q(t), 1) B (p(t), q(t), 1)^T \\
   &= (t^2, t, 1) K B K^T (t^2, t, 1)^T \\
   &= (t^2, t, 1) 
\begin{bmatrix}
a & b & d \\
b & c & e \\
d & e & f
\end{bmatrix}
(t^2, t, 1)^T \\
   &= a t^4 + 2 b t^3 + (c+2d)t^2 + 2et + f
.\end{align*}
Zgornja enakost velja natanko tedaj, ko je $a=b=e=f=0$ in $c=-2d$. Tedaj je matrika $KBK^T$ oblike
$$
\begin{bmatrix}
0 & 0 & d \\
0 & -2d & 0 \\
d & 0 & 0
\end{bmatrix}
=
dD.
$$
\end{dokaz}

%%%%%%%%%%%%%%%%%%%%%%%%%%
Zgornja trditev ima dve koristni posledici. Najprej bomo dokazali, da lahko vsako parabolično krivuljo parametriziramo s kvadratično parametrizacijo, kasneje pa še, da različni matriki iz množice $\PP$ ne porodita nujno različnih paraboličnih krivulj.

\begin{posledica}\label{vsakaparabola}
Vsaka parabolična krivulja ima kvadratično parametrizacijo.
\end{posledica}

% NOVO %%%%%%%%%%%%%%%%%%%%%%%%%%%%
\begin{dokaz}
Res, saj smo parabolično krivuljo definirali kot
$$ C_B = \{ (x,y); (x,y,1) B (x, y, 1)^T = 0, B \in \PP \}.$$
Matrika $B$ je iz množice $\PP$, ki je definirana kot množica matrik, ki so afino podobne $D$. Matriko $B$ torej gotovo lahko zapišemo kot $ABA^T$ za neko afino matriko $A$. Po zgornji trditvi ima parabolična krivulja $C_B$ kvadratično parametrizacijo.
\end{dokaz}
%%%%%%%%%%%%%%%%%%%%%%%%%%%%%

% NOVO %%%%%%%%%%%%%%%%%%%%%%%%%%%%

\begin{posledica}\label{bb*}
Naj bosta matriki $B$ in $\widetilde{B}$ elementa množice $\PP$. Enakost $C_B = C_{\widetilde{B}}$ velja natanko tedaj, ko velja $B = d \widetilde{B}$, $d \neq 0$.
\end{posledica}

\begin{dokaz}
%Dokaz je enostaven. 
Upoštevamo definicijo parabolične krivulje in hitro vidimo:
\begin{align*}
C_B    &= \{ (x,y); (x,y,1) B (x, y, 1)^T = 0\} \\
	&= \{ (x,y); (x,y,1) d \widetilde{B} (x, y, 1)^T = 0\} \\
	&= \{ (x,y); (x,y,1) \widetilde{B} (x, y, 1)^T = 0\} \\
	&= C_{\widetilde{B}}.
\end{align*}
\end{dokaz}
%%%%%%%%%%%%%%%%%%%%%%%%%%%%%

Ali lahko vsako trojico različnih točk interpoliramo s parabolično krivuljo? 

\begin{trditev}
Različnih kolinearnih točk $T_0, T_1, T_2$ ne moremo interpolirati s parabolično krivuljo.
\end{trditev}

% NOVO %%%%%%%%%%%%%%%%%%%%%%%%%%%%

% Ta dokaz ni dobro napisan. Razmisli.

\begin{dokaz}
Vsako parabolično krivuljo lahko parametriziramo s $(t^2, t,1)K$ za neko afino matriko $K$. Izberimo tri različne vrednosti parametra $t$, iz katerih dobimo tri linearno neodvisne trojice $(t^2, t, 1)$. Ko vsako od teh trojic pomnožimo z matriko $K$, dobimo tri točke na paraboli, ki so gotovo nekolinearne, ker je $K$ afina matrika. Pokazali smo, da ne za nobene tri vrednosti parametra $t$ ne dobimo kolinearnih točk, torej ne moremo najti parabolične krivulje, ki bi potekala skozi kolinearne točke.
\end{dokaz}
%%%%%%%%%%%%%%%%%%%%%%%%%%%%%

Imejmo sedaj tri nekolinearne točke v ravnini $T_0, T_1, T_2$. Iščemo parabolično krivuljo $(p, q)$, ki bo pri nekih parametrih $t_0, t_1, t_2$ interpolirala dane točke, to je % JE TO PRAV ???
$$ T_0 = (p(t_0), q(t_0)), \qquad T_1 = (p(t_1), q(t_1)), \qquad T_2 = (p(t_2), q(t_2)).$$
Za parametre $t_0, t_1, t_2$ lahko zahtevamo $t_0 < t_1 < t_2$ in še dodatno $t_0 = 0$ in $t_2 = 1$. Za poljuben $t_1 \in (0,1)$  lahko najdemo taka $p(t), q(t)$ stopnje največ dve, da bo krivulja $ \{(p, q), t \in [0,1] \}$ rešila naš interpolacijski problem. Označimo $t_1$ z $\al.$  Polinoma $p$ in $q$  lahko dobimo z reševanjem sistema enačb
\begin{equation}\label{sistem}
T_0 = (p(0), q(0)), \qquad T_1 = (p(\al), q(\al)), \qquad T_2 = (p(1), q(1)).
\end{equation}
Omenimo še, da je z izbiro $\al$ interpolacijska krivulja natanko določena, saj je zgornji sistem linearni sistem šestih enačb za šest neznank (koeficienti polinomov $p$ in $q$).

Pokazali bomo, da lahko kvadratično parametrizacijo $(p(t), q(t),1)$ dobimo tudi drugače. Pred tem definirajmo še Vandermondovo matriko za naš primer, torej za parametre $t_0 = 0, t_1 = \al$ in $t_2 = 1$, in konfiguracijsko matriko za dane točke.

\begin{definicija}
Za realno število $\al$ definiramo \emph{Vandermondovo matriko} $V(\al)$,
$$V(\al) = 
\begin{bmatrix}
0 & 0 & 1 \\
\al ^2 & \al & 1 \\
1 & 1 & 1
\end{bmatrix}
.$$
Za točke v ravnini $T_0, T_1, T_2$ definiramo konfiguracijsko matriko
$$R(T_0, T_1, T_2) = 
\begin{bmatrix}
T_0 & 1 \\
T_1 & 1 \\
T_2 & 1
\end{bmatrix}
.$$
\end{definicija}


\begin{trditev}\label{parametrizacija}
Naj bodo $T_0, T_1, T_2$  nekolinearne točke. Enolična kvadratična parametrizacija $(p, q)$, ki zadošča sistemu \eqref{sistem}, je podana z naslednjim predpisom:
$$ \Phi_\al(t; T_0, T_1, T_2) = (t^2, t, 1) V(\al)^{-1} R(T_0, T_1, T_2).$$
\end{trditev}


% NOVO %%%%%%%%%%%%%%%%%%%%%%%%%%%%
\begin{opomba}
Inverz Vandermondove matrike v eksplicitni obliki je
$$
V(\al)^{-1} = 
\begin{bmatrix}
\frac{1}{\al} & \frac{1}{\al^2 - \al} & \frac{1}{1-\al} \\
- \frac{\al + 1}{\al} & \frac{1}{\al - \al^2} & \frac{\al}{\al-1} \\
1 & 0 & 0
\end{bmatrix}
.$$
\end{opomba}
%%%%%%%%%%%%%%%%%%%%%%%%%%%%%

% NOVO %%%%%%%%%%%%%%%%%%%%%%%%%%%%
\begin{dokaz}
Označimo najprej $T_0 = (x_0, y_0), T_1 = (x_1, y_1), T_2 = (x_2, y_2)$. Računamo:
%$$
\begin{align*}
%%$$
\Phi_\al(t; T_0, T_1, T_2) &=  (t^2, t, 1) V(\al)^{-1} R(T_0, T_1, T_2) \\
				&=  (t^2, t, 1)  
\begin{bmatrix}
\frac{1}{\al} & \frac{1}{\al^2 - \al} & \frac{1}{1-\al} \\
- \frac{\al + 1}{\al} & \frac{1}{\al - \al^2} & \frac{\al}{\al-1} \\
1 & 0 & 0
\end{bmatrix}
\begin{bmatrix}
x_0 & y_0 & 1 \\
x_1 & y_1 & 1 \\
x_2 & y_2 & 1 
\end{bmatrix}
\\
				&= \frac{1}{\al^2 - \al} 
\begin{bmatrix}
t^2(\al - 1) - t(\al^2 - 1) + \al(\al - 1) \\
 t^2 - t \\
\al t(\al - t)
\end{bmatrix}
^T
\begin{bmatrix}
x_0 & y_0 & 1 \\
x_1 & y_1 & 1 \\
x_2 & y_2 & 1 
\end{bmatrix}
.\\
\end{align*}
Ko rezultat uredimo, dobimo naslednje: 
$$ 
\begin{bmatrix}
t^2 ( \frac{x_0}{\al} + \frac{x_1}{\al^2 - \al} - \frac{x_2}{\al -1} ) + t(- \frac{x_0(\al + 1)}{\al} - \frac{x_1}{\al^2 - \al} + \frac{x_0 \al}{\al - 1}) + x_0 \\
t^2 ( \frac{y_0}{\al} + \frac{y_1}{\al^2 - \al} - \frac{y_2}{\al -1} ) + t(- \frac{y_0(\al + 1)}{\al} - \frac{y_1}{\al^2 - \al} + \frac{y_0 \al}{\al - 1}) + y_0 \\
1
\end{bmatrix}
^T
= (p(t), q(t), 1).
$$
Preverimo, da polinom $p$ ustreza trditvi, torej da interpolira točke $T_0, T_1, T_2$. Res:
\begin{align*}
p(0) &= x_0, \\
p(\al) &= \al x_0 + \frac{\al x_1}{\al - 1} - \frac{\al^2 x_2}{\al - 1} - x_0(\al + 1) - \frac{x_1}{\al - 1} + \frac{\al^2 x_2}{\al - 1} + x_0 = x_1, \\
p(1) &= \frac{x_0}{\al} + \frac{x_1}{\al^2 - \al} - \frac{x_2}{\al -1} - \frac{x_0(\al + 1)}{\al} - \frac{x_1}{\al^2 - \al} + \frac{x_0 \al}{\al - 1} + x_0 = x_2.
\end{align*}
Analogno velja za polinom $q$, torej smo res našli kvadratično parametrizacijo interpolacijske krivulje.
\end{dokaz}
%%%%%%%%%%%%%%%%%%%%%%%%%%%%%

Definirajmo še dve matriki in množico, ki jih bomo potrebovali za dokaz glavnega izreka nekoliko kasneje.

\begin{definicija}\label{aalfa}
Definiramo matriki

$$A_{\al} = V(\al) \ D \ V(\al)^T $$
in
$$B_{\al} =  R(T_0, T_1, T_2)^{-1}\  A_{\al} \  (R(T_0, T_1, T_2)^{-1})^T.$$
Za nekolinearne točke $T_0, T_1, T_2$ naj bo $\B(T_0, T_1, T_2)$ množica vseh paraboličnih krivulj, ki potekajo skozi dane točke.

\end{definicija}

\begin{opomba}
Matriko $A_\al$ enostavno izračunamo in zapišemo eksplicitno z % JE z RES OK?
$$A_\al = 
\begin{bmatrix}
0 & \al^2 & 1 \\
\al^2 & 0 & (\al - 1)^2 \\
1 & (\al -1)^2 & 0
\end{bmatrix}
.$$

\end{opomba}

Naslednja trditev pokaže, da obstaja bijekcija med zgoraj definirano množico $\B(T_0, T_1, T_2)$ in množico $\R - \{0, 1\}$.

\begin{trditev}
Za nekolinearne točke $(T_0, T_1, T_2)$ je preslikava, ki $\al$ priredi matriko $B_{\al}$, bijekcija med $\R - \{0, 1\}$ in množico $\B(T_0, T_1, T_2)$.
\end{trditev}

% NOVO %%%%%%%%%%%%%%%%%%%%%%%%%%%%


%\begin{opomba}
%V nadaljevanju bomo zaradi jasnosti uporabjlali krajše oznake, kadar bo iz konteksta jasno, katere so točke $T_0, T_1, T_2$. 
%\end{opomba}


\begin{dokaz}
Iz trditve \ref{parametrizacija} sledi, da za vsak $\al \in \R - \{0,1 \}$ parametrizacija $\Phi_\al(t)$ parametrizira neko parabolo $C_B$ iz množice $\B(T_0, T_1, T_2)$.
Obratno, zaradi posledice \ref{vsakaparabola} velja, da za vsako parabolo iz $\B(T_0, T_1, T_2)$ obstaja kvadratična parametrizacija, označimo jo s $\Psi(t)$.
Če za parametrizacijo $\Psi(t)$ velja, da je $\Psi(t_i) = (T_i, 1)$, postavimo $\al = \frac{t_1 - t_0}{t_2 - t_0}$.
Opazimo, da velja $\Phi_\al(t) = \Psi(t_0 + (t_2 - t_0)t)$ 
%. Od tod sledi
%\begin{align*}
%\Phi_\al(t_0) = \Psi(t_0 + (t_2 - t_0)t_0
, torej tudi $\Phi_\al$ parametrizira parabolo $C_B$. Sledi, da je $C_B$ element množice $\B(T_0, T_1, T_2)$ natanko tedaj, ko jo lahko parametriziramo s $\Phi_\al$ za nek $\al \in \R - \{0,1 \}$.
\end{dokaz}

% Ojoj !!!!!!!!!!!!!!!!!!!!!!!!!!!!!!!!!!!!!!!!!

% To

% tu

% moram

% še

% dobro

% razmislit.

%%%%%%%%%%%%%%%%%%%%%%%%%%%%%


Za dane nekolinearne točke $T_0, T_1, T_2$ lahko vpeljemo \emph{poševni koordinatni sistem} tako, da za neko četrto točko $T_3$ obstaja vektor $p = (p_0, p_1, p_2)$, da velja
% SPREMENJENO; DODANA VSOTA
\begin{align*}
(T_3, 1) &= \sum_{i=0}^{2} p_i (T_i, 1) \\
	   & = p R(T_0, T_1, T_2).
	   \end{align*}
% KONEC SPREMEMBE 
Sledi $ p = (T_3, 1) R(T_0, T_1, T_2)^{-1}$.

Nekaj lastnosti tako definiranega vektorja $p$ podaja spodnja trditev.

\begin{trditev}\label{vektorp}
Za zgoraj definiran $p$ velja $p_0 + p_1 + p_2 = 1$ in $p_i = 0$ natanko tedaj, ko točka $T_3$ leži na isti premici kot točki $T_j$ in $T_k$, $j, k \in \{0, 1, 2 \}$. 
\end{trditev}

% NOVO %%%%%%%%%%%%%%%%%%%%%%%%%%%%
\begin{dokaz}

Dovolj je dokazati, da velja $p_0 = 0$ in $p_2 = 1- p_1$ natanko tedaj, ko točka $T_3$ leži na isti premici kot točki $T_1$ in $T_2$.
Vektor $p$ je v tem primeru enak $(0, p_1, 1-p_1)$. Glede na definicijo poševnega koordinatnega sistema velja
$$(T_3, 1) = (x_3, y_3, 1) = (p_1 x_1 + (1-p_2)x_2, p_1 y_1 + (1- p_2) y_2, 1).$$
Premica, ki poteka skozi točki $T_1$ in $T_2$, je podana z enačbo
$$ y = \frac{y_2 - y_1}{x_2 - x_1} x + \frac{x_2 y_1 - x_1 y_2}{x_2 - x_1} .$$
Z enostavnim računom se prepričamo, da točka $T_3$ leži na tej premici.

\end{dokaz}
%%%%%%%%%%%%%%%%%%%%%%%%%%%%%

Označimo sedaj s $T = \{T_0, T_1, T_2, T_3 \}$ nabor štirih točk v ravnini, od katerih nobene tri niso kolinearne. Take točke so oglišča konveksnega štirikotnika, če nobena točka $T_i$ ni v konveksni lupini preostalih treh točk.

\begin{trditev}\label{konveks}
Točke iz $T$ so oglišča konveksnega štirikotnika natanko tedaj, ko velja $p_0 p_1 p_2 < 0$, kjer so $p_0, p_1, p_2$ komponente vektorja $p$, za katerega velja $(T_3, 1) = p R(T_0, T_1, T_2).$
\end{trditev}

% NOVO %%%%%%%%%%%%%%%%%%%%%%%%%%%%
\begin{dokaz}

% Ne znam.

\end{dokaz}
%%%%%%%%%%%%%%%%%%%%%%%%%%%%%

Sedaj lahko zapišemo glavni izrek prvega poglavja.

\begin{izrek}
Naj bo $T = \{ T_0, T_1, T_2, T_3 \}$ nabor štirih točk v ravnini, od katerih nobene tri niso kolinearne.

\begin{enumerate}[i)]

\item Če so točke iz $T$ oglišča konkavnega štirikotnika, danih točk ne moremo interpolirati s parabolično krivuljo.

\item Če so točke iz $T$ oglišča paralelograma, danih točk ne moremo interpolirati s parabolično krivuljo.

\item Če so točke iz $T$ oglišča trapeza, ki ni paralelogram, lahko dane točke interpoliramo z natanko eno parabolično krivuljo.

\item Če so točke iz $T$ oglišča konveksnega štirikotnika, ki ni trapez, lahko dane točke interpoliramo z natanko dvema paraboličnima krivuljama.
\end{enumerate}

\end{izrek}

\begin{dokaz}
Matriki $A_\al$ in $B_\al$ smo definirali v \ref{aalfa}. Četrta točke $T_3$ leži na neki parabolični krivulji iz množice  $\B(T_0, T_1, T_2)$ natanko tedaj, ko obstaja tak $\al \in \R - \{0,1 \}$, da točka $T_3$ leži na $C_{B_\al}$. Sledi
\begin{align}
0	&= (T_3, 1) B_\al (T_3, 1)^T \nonumber \\
	&= (T_3, 1) R(T_0, T_1, T_2)^{-1}\  A_{\al} \  (R(T_0, T_1, T_2)^{-1})^T (T_3, 1)^T \nonumber \\
	&= (p_0, p_1, p_2) A_\al (p_0, p_1, p_2)^T \nonumber \\
	&= \al^2 p_0 p_1 + (\al - 1)^2 p_1 p_2 + p_0 p_2 \label{14} \\ % Spustimo 2 - lahko izpostavimo iz vseh členov
	&= \al^2 p_1(p_0 + p_2) - 2 \al p_1 p_2 + p_2(p_0 + p_1). \label{15}
\end{align}
Dobili smo kvadratno enačbo za $\al$, katere diskriminanta je 
\begin{align*}
D	&= 4 p_1^2 p_2^2 - 4 p_1 p_2(p_0 + p_2)(p_0 + p_1)  \\
	&= 4 p_1 p_2(p_1 p_2 - (1 - p_1)(1 - p_2))  \\
	&= 4 p_1 p_2(p_1 p_2 - 1 + p_1 + p_2 - p_1 p_2)  \\
	&= - 4 p_0 p_1 p_2. 
\end{align*}
Upoštevali smo, da velja $ p_0 + p_1 + p_2 = 1$. Produkt $p_0 p_1 p_2$ je različen od nič, saj nobene tri točke niso kolinearne. Opazimo naslednje:
\begin{enumerate}[a)]
\item enačba \eqref{15} ima enolično rešitev natanko tedaj, ko je $p_1 = 1$,
\item $\al = 0$ je ena od rešitev enačbe \eqref{15} natanko tedaj, ko je $p_2 = 1$, % druga rešitev je \al = \frac{2}{p_0 + 1}
\item $\al = 1$ je ena od rešitev enačbe \eqref{14} natanko tedaj, ko je $p_0 = 1$, % druga rešitev je \al = \frac{p_2 - 1}{p_2 + 1}
\end{enumerate}

Če sta dva od parametrov $p_0, p_1, p_2$ enaka $1$, je rešitev enačbe \eqref{15} bodisi $\al = 0$ bodisi $\al = 1$. V tem primeru ne obstaja interpolacijska parabolična krivulja za dane točke. Izkaže se, da so tem primeru točke iz $T$ oglišča paralelograma.

Če je natanko en od parametrov $p_0, p_1, p_2$ enak $1$, ima enačba \eqref{15} natanko eno rešitev. Natančneje, 
\begin{itemize}
\item če je $p_0 = 1$, je rešitev $\al = \frac{p_1 + 1}{p_1 - 1} = \frac{p_2 - 1}{p_2 + 1}$,
\item če je $p_1 = 1$, je rešitev $\al = \frac{p_0 + 1}{2}$,
\item če je $p_2 = 1$, je rešitev $\al = \frac{2}{p_0 + 1} = \frac{2}{1-p_2}$.
\end{itemize}
V tem primeru so točke iz $T$ oglišča trapeza, ki ni paralelogram.

Če so vse vrednosti parametrov $p_0, p_1, p_2$ različne od $1$, točke iz $T$ niso oglišča trapeza. Iz posledice \ref{konveks} sledi, da so točke iz $T$ oglišča konveksnega štirikotnika natanko tedaj, ko je $p_0 p_1 p_2 < 0$, kar pomeni, da je diskriminanta enačbe \eqref{15} strogo pozitivna in obstajata dve različni rešitvi. To sta
\begin{equation*}
\al_{1,2} = \frac{p_1 p_2 \pm \sqrt{p_0 p_1 p_2}}{p_1 ( p_0 + p_2)}.
\end{equation*}
 Tedaj lahko točke iz $T$ interpoliramo z dvema različnima paraboličnima krivuljama. 

Enačba \eqref{15} nima rešitve, če je njena diskriminanta negativna. Kot zgoraj je to natanko tedaj, ko so točke iz $T$ oglišča konkavnega štirikotnika.
\end{dokaz}

Konkreten interpolacijski problem rešimo tako, da izračunamo vektor $p$ iz trditve \ref{vektorp}, iz njegovih komponent pa izračunamo ustrezen $\al$. Parametrizacijo krivulje -- torej polinoma $p$ in $q$ -- smo eksplicitno zapisali v dokazu trditve \ref{parametrizacija}, za implicitno obliko parabolične krivulje $C_{B_\al}$ pa izračunamo še matriki $A_\al$ in $B_\al$ iz definicije \ref{aalfa}.

%%%%%%%%%%%%%%%%%%%%%%%%%%%%%%%%%%%%%%%%%%%%%%%%%%%%%%%%%%%%%%%%%%%%%%%%%%%%%%%%%
% --------------------------------------------------------------------------------------------------------------------------------------------------------------------------------------------------------------------------
%%%%%%%%%%%%%%%%%%%%%%%%%%%%%%%%%%%%%%%%%%%%%%%%%%%%%%%%%%%%%%%%%%%%%%%%%%%%%%%%%


\section{Interpolacija z Lagrangeevimi baznimi polinomi}

V tem poglavju si bomo ogledali rešitev interpolacijskega problema na bolj algebraičen način. Vemo, da lahko štiri točke interpoliramo s kubično krivuljo. Interpolacijsko krivuljo bomo zapisali z Lagrangeevimi baznimi polinomi, kar bo enostavneje, kot če bi delali v standardni bazi. Definirajmo najprej interpolacijski problem.

Za dane točke $T_i$ v ravnini, $i = 0, 1, 2, 3$, iščemo parabolično krivuljo $\textbf{p} : [ 0, 1 ] \rightarrow \R^2$, da bo veljalo $$\textbf{p}(t_i) = T_i, \qquad i = 0, 1, 2, 3$$ 

Dodatno zahtevamo, da so parametri urejeni, torej $t_0 < t_1 < t_2 < t_3$. Brez škode za splošnost lahko postavimo $t_0 = 0$ in $t_3 = 1$.

Definirajmo najprej Lagrangeeve bazne polinome v splošnem, kasneje pa bomo konkretno zapisali tiste, ki jih bomo potrebovali pri reševanju našega problema, torej polinome stopnje tri.


\begin{definicija}
Lagrangeeve bazne polinome stopnje $n$ definiramo

$$ \ell_{i,n}(t) = \prod_{\substack{j=0 \\ j \neq i}}^{n} \frac{t - t_j}{t_i - t_j} , \qquad i = 0, 1, \ldots n.$$	
\end{definicija}

Lagrangeevi bazni polinomi so vsi stopnje $n$. Poleg tega zanje velja $$\ell_{i,n}(t_j) = \delta_{ij}.$$
Oglejmo si polinome $ \ell_{0,3}(t), \ell_{1,3}(t), \ell_{2,3}(t)$ in $ \ell_{3,3}(t) $, ki so baza za prostor polinomov tretje stopnje:

\begin{align*}
	\ell_{0,3}(t) &= \frac{(t - t_1)(t - t_2)(t - t_3)}{(t_0 - t_1)(t_0 - t_2)(t_0 - t_3)}, \\
	\ell_{1,3}(t) &= \frac{(t - t_0)(t - t_2)(t - t_2)}{(t_1 - t_0)(t_1 - t_2)(t_1 - t_3)}, \\
	\ell_{2,3}(t) &= \frac{(t - t_0)(t - t_1)(t - t_3)}{(t_2 - t_0)(t_2 - t_1)(t_2 - t_3)}, \\
	\ell_{3,3}(t) &= \frac{(t - t_0)(t - t_1)(t - t_2)}{(t_3 - t_0)(t_3 - t_1)(t_3 - t_2)}. 	
\end{align*}

Interpolacijsko krivuljo lahko razvijemo po Lagrangeevih baznih polinomih kot 
\begin{equation}\label{razvoj}
p(t) = \sum_{i=0}^{3} a_i \ell_{i,n}(t),
\end{equation}

kjer so $a_i \in \R$ neznani koeficienti.

Hitro lahko vidimo, kaj so koeficienti $a_i$ pri interpolacijskem polinomu. Uvodoma smo zapisali, da naj za $\textbf{p}$ velja, da je $\textbf{p}(t_j) = T_j$. Ko vstavimo $t_j$ v \eqref{razvoj}, dobimo
\begin{align*}
\textbf{p}(t_j) &=  \sum_{i=0}^{3} a_i \ell_{i,n}(t_j) \\
	 &= \sum_{i=0}^{3} a_i \delta_{ij} \\
	&= a_j
,\end{align*}
od koder sledi, da so iskani koeficienti kar točke, ki jih interpoliramo. Interpolacijski polinom v Lagrangeevi obliki torej zapišemo kot 
$$\textbf{p}(t) = \sum_{i=0}^{3} T_i \ell_{i,n}(t).$$

Spomnimo, da so polinomi $\ell_{i,n}$ stopnje tri. Ker iščemo parabolično krivuljo, mora biti vodilni koeficient enak nič. To nam da nelinearen sistem
\begin{equation}\label{vodkoef}
\sum_{i = 0}^{3} \frac{T_i}{\prod_{\substack{j = 0 \\ j \neq i}}^3(t_i - t_j)} = 0
\end{equation}

za neznana parametra $t_1$ in $t_2$, enačbi pa dobimo iz komponent \eqref{vodkoef}, saj so $T_i$ točke v ravnini.


%%%%%%%%%%%%%%%%%%%%%%%%%%%%%%%%%%%%%%%%%%%%%%%%%%%%%%%%%%%%%%%%%%%%%%%%%%%%%%%%%
% --------------------------------------------------------------------------------------------------------------------------------------------------------------------------------------------------------------------------
%%%%%%%%%%%%%%%%%%%%%%%%%%%%%%%%%%%%%%%%%%%%%%%%%%%%%%%%%%%%%%%%%%%%%%%%%%%%%%%%%


\section{Hermitov primer}

Nazadnje si oglejmo še naslednji interpolacijski problem. Za dani različni točki $T_0$ in $T_1$ v ravnini in dani smeri $d_0, d_1, \lVert d_0 \rVert = \lVert d_1 \rVert = 1,$ želimo najti parabolično krivuljo $\textbf{p} : [ 0, 1 ] \rightarrow \R^2,$ da bo veljalo
\begin{align*}
\textbf{p}(0) = T_0&, \qquad \textbf{p}(1) = T_1, \\
\textbf{p}'(0) = d_0 d_0&, \qquad \textbf{p}'(1) = d_1 d_1 
\end{align*}

za skalarja $d_0, d_1 > 0$.

Podobno kot v prejšnjem poglavju bomo zapisali interpolacijsko krivuljo stopnje tri in določili taki $d_0$ in $d_1$ tako, da bo vodilni koeficient enak nič. Na ta način bomo dobili parabolično krivuljo. Interpolacijsko krivuljo zapišemo s pomočjo deljenih diferenc:

\begin{align*}
\textbf{p}(t) &= \textbf{p}(0) + t [0,0]\textbf{p} + t^2[0,0,1]\textbf{p} + t^2(t - 1)[0,0,1,1]\textbf{p} \\
	&= T_0 + t d_0 + t^2(T_1 - t_0 - d_o) + t^2(t - 1)(d_0 + d_1 - 2(T_1 - T_0))
.\end{align*}

Rešiti moramo torej enačbo

\begin{equation}
d_0 + d_1 - 2(T_1 - T_0) = 0
.\end{equation}

%%%%%%%%%%%%%%%%%%%%%%%%%%%%%%%%%%%%%%%%%%%%%%%%%%%%%%%%%%%%%%%%%

\section*{Slovar strokovnih izrazov}

\geslo{}{}
\geslo{}{}


% seznam uporabljene literature
\begin{thebibliography}{99}

%\bibitem{}

\end{thebibliography}

\end{document}

