\documentclass{beamer}

\usepackage[slovene]{babel}
\usepackage{amsfonts,amssymb}
\usepackage[utf8]{inputenc}
\usepackage{lmodern}
\usepackage[T1]{fontenc}

\usetheme{Warsaw}

\def\N{\mathbb{N}} % mnozica naravnih stevil
\def\Z{\mathbb{Z}} % mnozica celih stevil
\def\Q{\mathbb{Q}} % mnozica racionalnih stevil
\def\R{\mathbb{R}} % mnozica realnih stevil
\def\C{\mathbb{C}} % mnozica kompleksnih stevil


\def\qed{$\hfill\Box$}   % konec dokaza
\newtheorem{izrek}{Izrek}
\newtheorem{trditev}{Trditev}
\newtheorem{posledica}{Posledica}
\newtheorem{lema}{Lema}
\newtheorem{definicija}{Definicija}
\newtheorem{pripomba}{Pripomba}
\newtheorem{primer}{Primer}
\newtheorem{zgled}{Zgled}
\newtheorem{zgledi}{Zgledi uporabe}
\newtheorem{zglediaf}{Zgledi aritmetičnih funkcij}
\newtheorem{oznaka}{Oznaka}

\title{Geometrijska interpolacija štirih točk s parabolično krivuljo}
\author{Tjaša Bajc}
\institute{mentorica \\ izr.~prof.~dr.~Marjetka Knez}

%\date{24.\ februar 2017}

\begin{document}


%%%%%%%%%%%%%%%%%%%%%%%%%%%%%%%%%%%%%%%%%%%%%%%%%%%%%%%%%%%%%%%%%%%%%

\begin{frame}
\titlepage
\end{frame}

%%%%%%%%%%%%%%%%%%%%%%%%%%%%%%%%%%%%%%%%%%%%%%%%%%%%%%%%%%%%%%%%%%%%%

\begin{frame}
\frametitle{Opis problema}

\end{frame}


%%%%%%%%%%%%%%%%%%%%%%%%%%%%%%%%%%%%%%%%%%%%%%%%%%%%%%%%%%%%%%%%%%%%%

\begin{frame}
\frametitle{Geometrijska interpolacija}

Opazujemo šritikotnik, ki ga tvorijo dane štiri točke. Od lastnosti tega štirikotnika je odvisno, ali točke lahko interpoliramo s parabolično krivuljo.

\begin{izrek}
Naj bo $P = \{ P_0, P_1,  P_2, P_3 \}$ nabor štirih točk, od katerih nobene tri niso kolinearne.

\begin{enumerate}[i)]
\item Če so točke iz $P$ oglišča konkavnega štirikotnika, danih točk ne moremo interpolirati s parabolo.
\item Če so točke iz $P$ oglišča paralelograma, danih točk ne moremo interpolirati s parabolo.
\item Če so točke iz $P$ oglišča trapeza, ki ni paralelogram, lahko dane točke interpoliramo z natanko eno parabolo.
\item Če so točke iz $P$ oglišča konveksnega štirikotnika, ki ni trapez, lahko dane točke interpoliramo z natanko dvema parabolama.
\end{enumerate}

\end{izrek}

\end{frame}


%%%%%%%%%%%%%%%%%%%%%%%%%%%%%%%%%%%%%%%%%%%%%%%%%%%%%%%%%%%%%%%%%%%%%

\begin{frame}
\frametitle{Lagrangeevi bazni polinomi}

\begin{definicija}

$$ \ell_{i,n}(t) = \prod_{\substack{j=0 \\  j \neq i}}^{n} \frac{t - t_j}{t_i - t_j}, \quad  i = 0, \ldots, n$$
\end{definicija}

\end{frame}

%%%%%%%%%%%%%%%%%%%%%%%%%%%%%%%%%%%%%%%%%%%%%%%%%%%%%%%%%%%%%%%%%%%%%

\begin{frame}
\frametitle{Razširitve teme}

\end{frame}

%%%%%%%%%%%%%%%%%%%%%%%%%%%%%%%%%%%%%%%%%%%%%%%%%%%%%%%%%%%%%%%%%%%%%



\end{document}